\documentclass[a4paper,oneside,12pt]{article}

\usepackage[slovene]{babel}    % slovenian language and hyphenation
\usepackage[utf8]{inputenc}    % make čšž work on input
\usepackage[T1]{fontenc}       % make čšž work on output
\usepackage[reqno]{amsmath}    % basic ams math environments and symbols
\usepackage{amssymb,amsthm}    % ams symbols and theorems
\usepackage{mathtools}         % extends ams with arrows and stuff
\usepackage{url}               % \url and \href for links
\usepackage{icomma}            % make comma a thousands separator with correct spacing
\usepackage{units}             % \unit[1]{m} and unitfrac
\usepackage{enumerate}         % enumerate style
\usepackage{array}             % mutirow
\usepackage[usenames]{color}   % colors with names
\usepackage{graphicx}          % images

\usepackage[bookmarks, colorlinks=true, linkcolor=black, anchorcolor=black,
  citecolor=black, filecolor=black, menucolor=black, runcolor=black,
  urlcolor=black, pdfencoding=unicode]{hyperref}  % clickable references, pdf toc
\usepackage[
  paper=a4paper,
  top=2.5cm,
  bottom=2.5cm,
  left=2.5cm,
  right=2.5cm
% textheight=24cm,
]{geometry}  % page geomerty

\newtheorem{izrek}{Izrek}
\newtheorem{posledica}{Posledica}

\theoremstyle{definition}
\newtheorem{definicija}{Definicija}
\newtheorem{opomba}{Opomba}
\newtheorem{zgled}{Zgled}

% basic sets
\newcommand{\R}{\ensuremath{\mathbb{R}}}
\newcommand{\N}{\ensuremath{\mathbb{N}}}
\newcommand{\Z}{\ensuremath{\mathbb{Z}}}
\renewcommand{\C}{\ensuremath{\mathbb{C}}}
\newcommand{\Q}{\ensuremath{\mathbb{Q}}}

% greek letters
\let\oldphi\phi
\let\oldtheta\theta
\newcommand{\eps}{\varepsilon}
\renewcommand{\phi}{\varphi}
\renewcommand{\theta}{\vartheta}

% vektorska analiza
\newcommand{\grad}{\operatorname{grad}}
\newcommand{\rot}{\operatorname{rot}}
\renewcommand{\div}{\operatorname{div}}

% lists with less vertical space
\newenvironment{itemize*}{\vspace{-1.5\parskip}\begin{itemize}\setlength{\itemsep}{0pt}\setlength{\parskip}{2pt}}{\end{itemize}\vspace{-1\parskip}}
\newenvironment{enumerate*}{\vspace{-1.5\parskip}\begin{enumerate}\setlength{\itemsep}{0pt}\setlength{\parskip}{2pt}}{\end{enumerate}\vspace{-1\parskip}}
\newenvironment{description*}{\vspace{-6pt}\begin{description}\setlength{\itemsep}{0pt}\setlength{\parskip}{2pt}}{\end{description}\vspace{-1\parskip}}

\newcommand{\Title}{NINDE 2016\,/\,17, 2.\ domača naloga}
\newcommand{\Author}{Jure Slak}
\title{\Title}
\author{\Author}
\date{\today}
\hypersetup{pdftitle={\Title}, pdfauthor={\Author}, pdfcreator={\Author},
            pdfproducer={\Author}, pdfsubject={}, pdfkeywords={}}  % setup pdf metadata

% \pagestyle{empty}              % vse strani prazne
\setlength{\parindent}{0pt}    % zamik vsakega odstavka
\setlength{\parskip}{10pt}     % prazen prostor po odstavku
\setlength{\overfullrule}{30pt}  % oznaci predlogo vrstico z veliko črnine

\begin{document}
\begin{center}
  \textbf{\large \Title} \\[12pt]
  Jure Slak, 27152005
\end{center}

\textbf{Naloga 1.} \\
Najprej je implementirana splošna eksplicitna RK metoda za reševanje sistemov
NDE v datoteki \verb|RKEplicit.m| s pomočjo nekaj pomožnih funkcij. V datoteki
\verb|make_RK4.m| je skripta, ki naredi shemo za RK4 in naredi objekt, ki
predstavjla solver po tej metodi. Podobno so v skripti \verb|make_CK.m| podatki
za obe RK metodi, na katerih bazira Cash-Karp metoda (ter tudi podatki za shemo
metode, predstavljene kot Algoritem 1.). Funkcija \verb|CashKarp| je
implementirana kot piše na nalogi. Če uporabimo algoritem 1, potem je pri
natančnosti $10^{-6}$ $h$konstantno enak $hmin$, kar je tudi logično, saj ocena
napake ni pravilna, in ne kaže dejanske napake. Če pa uporabimo dejansko
Cash-Karpovo metodo, pa se $h$ res prilagaja in tudi opravi veliko manj
izračunov. Celotna rešitev naloge je v \verb|nal1.m|. Prava Cash-Carp metoda,
kot je napisana v algoritmu 1, nikoli ne doseže vrednosti $x=2$, saj je do tja
premajhen korak. Ker ni specificirano, kaj naj naredimo v tem primeru, sem jaz
vseeno naredil majhen korak in izračunal vrednost $y(2)$.

\begin{verbatim}
CK-prava: y(2) = 1.9506227004274028
CK-alg1:  y(2) = 1.9684232501488073
RK4:      y(2) = 1.9506785300973337
\end{verbatim}

Pri implementaciji metod bi si lahko shranjevali že izračunane vrednosti $f$ in
s tem pospešili izvajanje.

\textbf{Naloga 2.} \\

Implementacija obeh metod je direktna, ena v datoteki
\verb|AdamsBashSistem.m| in druga v \verb|MilneSistem.m|.
Naloga je rešena v datoteki \verb|nal2.m|.

\begin{verbatim}
h = 0.1
Approx for y(10) using AB: -0.3498705377962865
Approx for y(10) using MI: -0.4414240151890469
h = 0.05
Approx for y(10) using AB: -0.4227331662395586
Approx for y(10) using MI: -0.4222779703683874
\end{verbatim}

\textbf{Naloga 3.} \\
Najprej rešimo nalogo z RK4 in iz slike preberemo približek za peto ničlo.
Nato s standardno tangentno metodo iščemo ničlo, pri čemer odvode in vrednosti
aproksimiramo z RK4, ki se od prejšnje ničle do naslednje sprehodi z maksimalnim
korakom $h$, ali pa z manjšim, če je celotna razdalja manjša. Metoda se lahko
sprehaja v levo ali v desno. Seveda pa je ničla bolj od tega kakšno natančnost
si zberemo pri Newtonovi iteraciji odvisna od začetnega $h$. Newtonova iteracija
pričakovano skonvergira v približno petih korakih tudi ob toleranci $10^{-16}$.
Za začetni približek sem vzel $7.5$.

\begin{verbatim}
h = 0.1
korak: 1, priblizek: 7.811265563327425
korak: 2, priblizek: 7.757021945794063
korak: 3, priblizek: 7.757430943970567
korak: 4, priblizek: 7.757430943793660
korak: 5, priblizek: 7.757430943793660
Nicla: 7.7574309437936604.

h = 0.01
korak: 1, priblizek: 7.811097555023491
korak: 2, priblizek: 7.756912740566238
korak: 3, priblizek: 7.757320650664874
korak: 4, priblizek: 7.757320650489378
korak: 5, priblizek: 7.757320650489378
Nicla: 7.7573206504893779.
\end{verbatim}


\end{document}
% vim: syntax=tex
% vim: spell spelllang=sl
% vim: foldlevel=99
% Latex template: Jure Slak, jure.slak@gmail.com

