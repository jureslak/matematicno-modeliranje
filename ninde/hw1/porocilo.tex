\documentclass[a4paper,oneside,12pt]{article}

\usepackage[slovene]{babel}    % slovenian language and hyphenation
\usepackage[utf8]{inputenc}    % make čšž work on input
\usepackage[T1]{fontenc}       % make čšž work on output
\usepackage[reqno]{amsmath}    % basic ams math environments and symbols
\usepackage{amssymb,amsthm}    % ams symbols and theorems
\usepackage{mathtools}         % extends ams with arrows and stuff
\usepackage{url}               % \url and \href for links
\usepackage{icomma}            % make comma a thousands separator with correct spacing
\usepackage{units}             % \unit[1]{m} and unitfrac
\usepackage{enumerate}         % enumerate style
\usepackage{array}             % mutirow
\usepackage[usenames]{color}   % colors with names
\usepackage{graphicx}          % images

\usepackage[bookmarks, colorlinks=true, linkcolor=black, anchorcolor=black,
  citecolor=black, filecolor=black, menucolor=black, runcolor=black,
  urlcolor=black, pdfencoding=unicode]{hyperref}  % clickable references, pdf toc
\usepackage[
  paper=a4paper,
  top=2.5cm,
  bottom=2.5cm,
  left=2.5cm,
  right=2.5cm
% textheight=24cm,
]{geometry}  % page geomerty

\newtheorem{izrek}{Izrek}
\newtheorem{posledica}{Posledica}

\theoremstyle{definition}
\newtheorem{definicija}{Definicija}
\newtheorem{opomba}{Opomba}
\newtheorem{zgled}{Zgled}

% basic sets
\newcommand{\R}{\ensuremath{\mathbb{R}}}
\newcommand{\N}{\ensuremath{\mathbb{N}}}
\newcommand{\Z}{\ensuremath{\mathbb{Z}}}
\renewcommand{\C}{\ensuremath{\mathbb{C}}}
\newcommand{\Q}{\ensuremath{\mathbb{Q}}}

% greek letters
\let\oldphi\phi
\let\oldtheta\theta
\newcommand{\eps}{\varepsilon}
\renewcommand{\phi}{\varphi}
\renewcommand{\theta}{\vartheta}

% vektorska analiza
\newcommand{\grad}{\operatorname{grad}}
\newcommand{\rot}{\operatorname{rot}}
\renewcommand{\div}{\operatorname{div}}

% lists with less vertical space
\newenvironment{itemize*}{\vspace{-1.5\parskip}\begin{itemize}\setlength{\itemsep}{0pt}\setlength{\parskip}{2pt}}{\end{itemize}\vspace{-1\parskip}}
\newenvironment{enumerate*}{\vspace{-1.5\parskip}\begin{enumerate}\setlength{\itemsep}{0pt}\setlength{\parskip}{2pt}}{\end{enumerate}\vspace{-1\parskip}}
\newenvironment{description*}{\vspace{-6pt}\begin{description}\setlength{\itemsep}{0pt}\setlength{\parskip}{2pt}}{\end{description}\vspace{-1\parskip}}

\newcommand{\Title}{NINDE 2016\,/\,17, 1.\ domača naloga}
\newcommand{\Author}{Jure Slak}
\title{\Title}
\author{\Author}
\date{\today}
\hypersetup{pdftitle={\Title}, pdfauthor={\Author}, pdfcreator={\Author},
            pdfproducer={\Author}, pdfsubject={}, pdfkeywords={}}  % setup pdf metadata

% \pagestyle{empty}              % vse strani prazne
\setlength{\parindent}{0pt}    % zamik vsakega odstavka
\setlength{\parskip}{10pt}     % prazen prostor po odstavku
\setlength{\overfullrule}{30pt}  % oznaci predlogo vrstico z veliko črnine

\begin{document}
\begin{center}
  \textbf{\large \Title} \\[12pt]
  Jure Slak, 27152005
\end{center}

\textbf{Naloga 1.} \\
Implementacija sestavljenega Simpsonovega in trapeznega pravila je zelo
direktna. Dvojni integral lahko izračunamo po Fubinijevem izreku kot
dvakratnega, na dva načina, ki sta matematično enaka, numerično pa se
razlikujeta v vrstnem redu seštevanja. Rezultat pri danih podatkih se še precej
razlikuje od točnega (121.9 proti 123.79). Če $m$ in $n$ povečujemo, metoda
konvergira, ne pa tudi, če povečujemo samo enega. Razlika med numeričnima
približkoma zaradi vrstnega reda integracije je v tem primeru zanemarljiva in
znaša okoli $10^{-14}$.

\begin{verbatim}
Zunanji po x: 123.7998130262172083
Zunanji po y: 123.7998130262171941
Točen:        121.9385660804582869
\end{verbatim}

\textbf{Naloga 2.} \\
Metoda uporablja sestavljeno Simpsonovo pravilo iz 1. naloge. Tretnutno ni
napisana paralelizabilno, saj pri deljenju intervala dovoljeno drugo napako
izračuna na podlagi prve, jo je pa mogoče paralelizirati. Tudi tukaj se
približki izboljšujejo, ko gre $\delta \to 0$ in napake ne presegajo dovoljene
napake $\delta$.

\begin{verbatim}
I =    1.9980118683563632
Točen: 1.9980009999997501
\end{verbatim}

Metodo bi se dalo pospešiti, tako da bi si shranjevala funkcijske vrednosti in
jih podajala naprej, saj se jih zelo veliko prekriva.

\textbf{Naloga 3.} \\
Metoda je implementirana malce splošneje za integracijo po kvadru $[a, b]
\subset \R^d$. Konvergenca je počasna, a zagotovljena po zakonu velikih števil.
Metodo bi bilo možno pospešiti z uporabo vektorizacije in več pomnilnika,
kjer bi si vnaprej generirali naključno matriko $n \times d$ s slučajnimi vektorji
in jih kasneje le uporabili. Poleg tega je metoda tudi zelo primerna za
paralelizacijo.

\begin{verbatim}
Točen:           I = 0.1594319846981149
Pri n =     100, I = 0.1544755419083274, err = 0.0049564427897875
Pri n =    1000, I = 0.1601708347878220, err = 0.0007388500897071
Pri n =   10000, I = 0.1602101986607355, err = 0.0007782139626206
Pri n =  100000, I = 0.1593783551198855, err = 0.0000536295782294
Pri n = 1000000, I = 0.1593951995388704, err = 0.0000367851592445
\end{verbatim}

\end{document}
% vim: syntax=tex
% vim: spell spelllang=sl
% vim: foldlevel=99
% Latex template: Jure Slak, jure.slak@gmail.com

